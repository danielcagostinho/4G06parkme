\documentclass[]{article}
\usepackage[margin=1.15in]{geometry}
%opening
\title{Park Me: Proof of Concept Contract}
\author{Daniel Agostinho\\agostd - 001414323\\ Group 5 \and Michael Bitzos\\bitzosm - 001405050\\ Group 5 \and Kathryn Brownlee\\brownlks - 001408416\\ Group 5  \and Anthony Chang\\changa7 - 001413615\\ Group 5 \and Ben Petkovsek\\petkovb - 001417104\\ Group 5}
\begin{document}
\date{November 25, 2018}
\maketitle


ParkMe is a mobile application that will assist drivers in locating and identifying available parking spaces in a parking lot. This application will also help businesses collect various statistics and trends about the parking spaces. For the demo Group 5 will present, they will promise that the following features/scenarios are presented by the end of Capstone 2018/2019:

\begin{itemize}
	\item Show that when a parking space is occupied, the application will show on the map that it is occupied.
	
	\item Allow users to view basic statistics about the parking lot (how many parking spaces get occupied at a certain hour, how long a parking space is occupied, etc.).
	
	\item Allow drivers to specify personalized settings for parking space preferences (handicap settings, etc).
	
	\item Allow drivers to navigate to the parking lot they specify and show a zoomed view of a mapping of the lot’s parking scheme.
\end{itemize}

The previous list of features are guaranteed to be shown at least once during the demonstration. They are considered the core functionalities Group 5’s ParkMe application and they feel that it will be a overall good demonstration of the product’s capabilities. While stretch goals listed in previous requirement documents are not promised, they are likely to appear. 
\end{document}
